\documentclass{scrartcl}

\usepackage{amsmath}
\usepackage{amssymb}
\usepackage{enumerate}
\usepackage[utf8]{inputenc}
\usepackage{braket}

\begin{document}

\section{Harmonischer Oszillator}

Gegeben sei ein quantenmechanischer harmonischer Oszillator beschrieben durch die eine Wellenfunktion
\[\psi(x,0)=\left(\frac{m\omega}{\pi \hbar}\right)^{1/4}\left(\frac{m\omega}{\sqrt 2 \hbar}x^2+\sqrt{\frac{3m\omega}{2\hbar}}x-\frac{1}{2\sqrt 2}\right)e^{-\frac{m\omega}{2\hbar}x^2}\]
Und den Eigenfunktionen
\begin{align*}
\psi_0(x)	&=\left(\frac{m\omega}{\pi\hbar}\right)^{1/4}e^{-\frac{m\omega}{2\hbar}x^2}	\\
\psi_1(x)	&=\left(\frac{m\omega}{\pi\hbar}\right)^{1/4}\sqrt{\frac{2m\omega}{\hbar}}xe^{-\frac{m\omega}{2\hbar}x^2}	\\
\psi_2(x)	&=\left(\frac{m\omega}{\pi\hbar}\right)^{1/4}\frac{1}{\sqrt 2}\left(\frac{2m\omega}{\hbar}x^2-1\right)e^{-\frac{m\omega}{2\hbar}x^2}
\end{align*}
und der Energie 
\[E_n =\left(n+\frac{1}{2}\right)\hbar\omega\]

\begin{enumerate}[a)]
\item Schreiben Sie die Wellenfunktion als eine Summe von Eigenfunktionen des harmonischen Oszillators und bestimmen Sie eine zeitabhängige Wellenfunktion $\psi(x,t)$.
\item[Lösung:]
Durch scharfes hinsehen wieht man:
\[\psi(x)=\frac{\sqrt 3}{2}\psi_1(x)+\frac{1}{2}\psi_2(x)\]
Der Faktor $\left(\frac{m\omega}{\pi\hbar}\right)^{1/4}e^{-\frac{m\omega}{2\hbar}x^2}$ kommt in allen Eigenfunktionen vor. Betrachte also nur die Klammer, $\psi_1$ ist die einzige Funktion die $x^1$ enthält und $\psi_2$ die einzige mit $x^2$.\\
Es gilt
\[\psi(x,t)=\sum_k N_k e^{-iE_k t/\hbar}\ket{k}=\sum_k N_k e^{-iE_k t /\hbar}\psi_k(x)\]
Es ist also
\begin{align*}
\psi(x,t)	&=N_0 e^{-iE_0 t/\hbar}\psi_0(x) + N_1 e^{-iE_1 t/\hbar}\psi_1(x) + N_2 e^{-iE_2 t/\hbar}\psi_2(x) \\
&= \frac{\sqrt 3}{2} e^{-\frac{3}{2}i\omega t}\psi_1(x) + \frac{1}{2} e^{-\frac{5}{2}i\omega t}\psi_2(x) \\
\end{align*}

\item Berechnen Sie Orts- und Impulsmittelwert $\braket{x}=\braket{\psi|x|\psi}$ und $\braket{p}=\braket{\psi|p|\psi}$ im Zustand $\psi(x,t)$
\item[Lösung:]
\begin{align*}
\braket{x}	&=\braket{\psi|x|\psi} = \frac 12 \sqrt{\frac{2\hbar}{m\omega}} \psi^\ast (\hat a + \hat a^\dagger) \psi \\
			&=\frac 12 \sqrt{\frac{2\hbar}{m\omega}}(N_0^\ast \bra 0 + N_1^\ast \bra 1 + N_2^\ast \bra 2)\:(\hat a +\hat a^\dagger)\:(N_0 \ket 0 + N_1 \ket 1 + N_2 \ket 2) \\
\intertext{Da die $N_i$ rein reell sind, ist $N_i=N_i^\ast$. Außerdem ist $N_0=0$}
			&= \frac 12 \sqrt{\frac{2\hbar}{m\omega}}\big(N_1 \bra 1 + N_2 \bra 2\big)(\hat a +\hat a^\dagger) \big(N_1 \ket 1 + N_2 \ket 2\big) \\
			&= \frac 12 \sqrt{\frac{2\hbar}{m\omega}}\big(N_1 \bra 1 + N_2 \bra 2\big)\:\hat a\big(N_1 \ket 1 + N_2 \ket 2\big) \\
			&+ \frac 12 \sqrt{\frac{2\hbar}{m\omega}}\big(N_1 \bra 1 + N_2 \bra 2\big)\:\hat a^\dagger\big(N_1 \ket 1 + N_2 \ket 2\big) \\
			&= \frac 12 \sqrt{\frac{2\hbar}{m\omega}}\big(N_1 \bra 1 + N_2 \bra 2\big)\big(N_1 \ket 0 + \sqrt 2 N_2  \ket 1\big) \\
			&+ \frac 12 \sqrt{\frac{2\hbar}{m\omega}}\big(N_1 \bra 1 + N_2 \bra 2\big)\big(\sqrt 2 N_1 \ket 2 + \sqrt 3 N_2 \ket 3\big) \\
			&= \frac 12 \sqrt{\frac{2\hbar}{m\omega}} \Big( N_1 N_1\braket{1|0} + \sqrt 2 N_1 N_2 \braket{1|1} + N_2 N_0 \braket{2|0} + \sqrt 2 N_2 N_2 \braket{2|1} \\
			&\qquad + \sqrt 2 N_1 N_1 \braket{1|2} + \sqrt 3 N_1 N_2 \braket{1|3} + \sqrt 2 N_2 N_1 \braket{2|2} + \sqrt 3 N_2 N_2 \braket{2|3}\Big)			
			\intertext{Wegen $\braket{i|j}=\delta_{i,j}$ vereinfacht sich der Term zu}
			&= \frac 12 \sqrt{\frac{2\hbar}{m\omega}}\Big(\sqrt 2 N_1 N_2 + \sqrt 2 N_2 N_1  \Big)	\\
			\intertext{$N_1=\frac{\sqrt 3}{2}, N_2=\frac{1}{2}$}
			&= \sqrt{\frac{2\hbar}{m\omega}}\Big(\sqrt 2 \frac{\sqrt 3}{2} \frac{1}{2}  \Big)	\\
			&= \sqrt{\frac{3\hbar}{2m\omega}}	\\
			&\Big(= \int_{-\infty}^{\infty} x |\psi(x)|^2\:dx\Big)
\end{align*}
Da $\braket{x}$ nicht von $t$ abhängt, ist
\begin{align*}
\braket{p}	&= \frac{d\braket{x}}{dt} \\
			&= 0
\end{align*}

\item Überprüfen Sie Heisenbergs Prinzip
\[(\Delta x)^2(\Delta p)^2\geq\frac{\hbar^2}{4}\]
\item[Lösung:]
Berechne zuerst $\braket{x^2}$ und $\braket{p^2}$:
\begin{align*}
\braket{x^2}	&= \braket{\psi|\hat x^2|\psi}		\\
				&= \frac 12 \sqrt{\frac{2\hbar}{m\omega}} \psi^\ast (\hat a + \hat a^\dagger)^2 \psi	\\
				&= \frac 12 \sqrt{\frac{2\hbar}{m\omega}} 
				(N_0^\ast\bra 0 + N_1^\ast \bra 1 + N_2^\ast \bra 2)
				\:(\hat a\hat a + 2 \hat a \hat a^\dagger + \hat a^\dagger\hat a^\dagger)\:
				(N_0\ket 0 + N_1 \ket 1 + N_2 \ket 2)		\\
				&= \frac 12 \sqrt{\frac{2\hbar}{m\omega}} 
				(N_1^\ast \bra 1 + N_2^\ast \bra 2)
				\:(\hat a\hat a + 2 \hat a \hat a^\dagger + \hat a^\dagger\hat a^\dagger)\:
				(N_1 \ket 1 + N_2 \ket 2)		\\
\end{align*}
\end{enumerate}

\section{Teilchen mit Spin im magnetischen Feld}
Betrachten Sie ein Elektron in einem Überlagerungszustand durch
\[\psi=\frac{1}{\sqrt 2}\begin{pmatrix}\frac{1+\sqrt 3}{2}\\\frac{1-\sqrt 3}{2}\end{pmatrix}\]

\begin{enumerate}[a)]
\item Bestimmen Sie die Wahrscheinlichkeit im Zustand $\psi$ Spin $\frac{\hbar}{2}$ in $x$- bzw. $y$-Richtung zu messen.
\item[Lösung:]
Wahrscheinlichkeit im Zustand $\psi$ den Spin $\frac\hbar 2$ in $x$-Richtung zu messen:
\begin{align*}
\braket{\psi|S_x|\psi}	
	&= 
	\left(\frac 1{\sqrt 2} \begin{pmatrix}\frac{1+\sqrt 3}{2} & \frac{1-\sqrt 3}{2}\end{pmatrix}\right)^\ast
	\frac \hbar 2 \begin{pmatrix} 0&1\\1&0 \end{pmatrix}
	\frac 1{\sqrt 2}\begin{pmatrix}\frac{1+\sqrt 3}{2}\\\frac{1-\sqrt 3}{2}\end{pmatrix}\\
	&= 	
	\frac \hbar 4 \begin{pmatrix}\frac{1+\sqrt 3}{2} & \frac{1-\sqrt 3}{2}\end{pmatrix}
	\begin{pmatrix} 0&1\\1&0 \end{pmatrix}
	\begin{pmatrix}\frac{1+\sqrt 3}{2}\\\frac{1-\sqrt 3}{2}\end{pmatrix}\\
	&=
	\frac \hbar {16} \begin{pmatrix}1+\sqrt 3 & 1-\sqrt 3\end{pmatrix}
	\begin{pmatrix}1-\sqrt 3\\1+\sqrt 3\end{pmatrix}\\
	&=
	\frac \hbar {16} 2 (1 + 3) = \frac \hbar 2
\end{align*}
Genauso für die $y$-Richtung:
\begin{align*}
\braket{\psi|S_y|\psi}	
	&= 
	\left(\frac 1{\sqrt 2} \begin{pmatrix}\frac{1+\sqrt 3}{2} & \frac{1-\sqrt 3}{2}\end{pmatrix}\right)^\ast
	\frac \hbar 2 \begin{pmatrix} 0&-i\\i&0 \end{pmatrix}
	\frac 1{\sqrt 2}\begin{pmatrix}\frac{1+\sqrt 3}{2}\\\frac{1-\sqrt 3}{2}\end{pmatrix}\\
	&= 	
	\frac \hbar 4 \begin{pmatrix}\frac{1+\sqrt 3}{2} & \frac{1-\sqrt 3}{2}\end{pmatrix}
	\begin{pmatrix} 0&-i\\i&0 \end{pmatrix}
	\begin{pmatrix}\frac{1+\sqrt 3}{2}\\\frac{1-\sqrt 3}{2}\end{pmatrix}\\
	&=
	\frac \hbar {16} \begin{pmatrix}1+\sqrt 3 & 1-\sqrt 3\end{pmatrix}
	\begin{pmatrix}-i+i\sqrt 3\\i+i\sqrt 3\end{pmatrix}\\
	&=
	\frac {i\hbar}{16} \begin{pmatrix}1+\sqrt 3 & 1-\sqrt 3\end{pmatrix}
	\begin{pmatrix}-1+\sqrt 3\\1+\sqrt 3\end{pmatrix}\\
	&=
	\frac {i\hbar}{16} (-1 + \sqrt 3 -\sqrt 3 + 3 + 1 + \sqrt 3  -\sqrt 3 -3) \\
	&=
	\frac{i\hbar}{16} 0 = 0
\end{align*}

\item Das Teilchen ruht in einem homogenen magnetischen Feld $\vec B = B_0 e_z$. Der Hamiltonoperator lautet
\[\hat H = -\vec \mu \vec B = \frac em \vec S \vec B \]
Berechnen Sie die Eigenvektoren und die Eigenwerte (Energien $E_1, E_2$) von $\hat H$. Ist $\psi$ ein stationärer Zustand? Begründen Sie!
\item[Lösung:]
\begin{align*}
\hat H &= \frac em \vec S \vec B = \frac em \vec S B_0 e_x = \frac em B_0 S_x = \frac em B_0 \frac \hbar 2 \begin{pmatrix}0&1\\1&0\end{pmatrix} \\
&= \frac{e\hbar B_0}{2m}\begin{pmatrix}0&1\\1&0\end{pmatrix}
\end{align*}
$\sigma_x = \begin{pmatrix}0&1\\1&0\end{pmatrix}$ hat die Eigenwerte $\lambda_1 = 1, \lambda_2 = -1$, also ist
\begin{align*}
E_1 &=  \frac{e\hbar B_0}{2m}\\
E_2 &= -\frac{e\hbar B_0}{2m}
\end{align*}
\end{enumerate}


\end{document}