\documentclass{scrartcl}

\usepackage{amsmath}
\usepackage{amssymb}
\usepackage{enumerate}
\usepackage[utf8]{inputenc}
\usepackage{braket}
\usepackage{tikz}

\newcommand{\Sx}{\begin{pmatrix}0&1\\1&0\end{pmatrix}}
\newcommand{\vecz}[2]{\begin{pmatrix}#1 \\ #2\end{pmatrix}}

\allowdisplaybreaks % allow breaking align enviornment over multiple pages

\begin{document}

\section{Doppelpendel}
Betrachten Sie ein gekoppeltes Pendel in einer Ebene mit den Massen $m_1$ und $m_2$ und den Seillängen $l_1$ und $l_2$ (siehe Abbildung:)
\begin{figure}
\begin{tikzpicture}
\draw (-5,0) -- (5,0);
\foreach \x in {-4.5,...,4} {
	\draw (\x,0) -- (\x+0.4,0.4);
}
\draw[thick] (0,0) -- (2,-3) node[xshift=0.4cm,midway] {$l_1$};
\draw[thick] (2,-3) -- (4,-5) node[xshift=0.4cm,midway] {$l_2$};
\draw(0,0) -- (0,-2);
\draw(2,-3) -- (2,-5);
\fill[thick,draw=black,fill=gray] (2,-3) circle (0.3cm) node[xshift=.6cm]{$m_1$};
\fill[thick,draw=black,fill=gray] (4,-5) circle (0.3cm) node[xshift=.6cm]{$m_2$};
\end{tikzpicture}
\end{figure}

\begin{enumerate}[a)]
\item Wie lautet die potentielle Energie des Systems im Schwerefeld (Schwerkraft auf eine Masse $m$: $F_z = -mg$).
\item[Lösung:] Potentielle Energie setzt sich aus $V_1$ und $V_2$ zusammen. Die Position der Massen sei mit $\vec r_1=\vecz{y_1}{z_1}$ bzw. mit $\vec r_2=\vecz{y_2}{z_2}$ bezeichnet ($z$-Achse zeige nach oben und $x$-Achse nach rechts).
\begin{align*}
V_1 &= -F_{1_z}z_1 \\
	&= m_1gz_1	\\
V_2 &= m_2gz_2 \\
V	&= V_1+V_2	\\
	&= m_1gz_1 + m_2gz_2\\
	&= g(m_1z_1 + m_2z_2)
\end{align*}

\item Schreiben Sie die Lagrangefunktion $L$ des Systems. Führen Sie Polarkoordinaten ein!
\item[Lösung:]
\begin{align*}
y_1 	&= l_1 \sin \phi_1	\\
x_1 	&= l_1 \cos \phi_1	\\
y_2		&= y_1 + l_2 \sin \phi_2	\\
		&= l_1 \sin \phi_1 + l_2 \sin \phi_2	\\
z_2		&= z_1 + l_2 \cos \phi_2	\\
		&= l_1 \cos \phi_1 + l_2 \cos \phi_2 
\end{align*}
Potentielle Energie ist aus a) bekannt. Berechne noch kinetische Energie:

\end{enumerate}




\section{Harmonischer Oszillator}

Gegeben sei ein quantenmechanischer harmonischer Oszillator beschrieben durch die eine Wellenfunktion
\[\psi(x,0)=\left(\frac{m\omega}{\pi \hbar}\right)^{1/4}\left(\frac{m\omega}{\sqrt 2 \hbar}x^2+\sqrt{\frac{3m\omega}{2\hbar}}x-\frac{1}{2\sqrt 2}\right)e^{-\frac{m\omega}{2\hbar}x^2}\]
Und den Eigenfunktionen
\begin{align*}
\psi_0(x)	&=\left(\frac{m\omega}{\pi\hbar}\right)^{1/4}e^{-\frac{m\omega}{2\hbar}x^2}	\\
\psi_1(x)	&=\left(\frac{m\omega}{\pi\hbar}\right)^{1/4}\sqrt{\frac{2m\omega}{\hbar}}xe^{-\frac{m\omega}{2\hbar}x^2}	\\
\psi_2(x)	&=\left(\frac{m\omega}{\pi\hbar}\right)^{1/4}\frac{1}{\sqrt 2}\left(\frac{2m\omega}{\hbar}x^2-1\right)e^{-\frac{m\omega}{2\hbar}x^2}
\end{align*}
und der Energie 
\[E_n =\left(n+\frac{1}{2}\right)\hbar\omega\]

\begin{enumerate}[a)]
\item Schreiben Sie die Wellenfunktion als eine Summe von Eigenfunktionen des harmonischen Oszillators und bestimmen Sie eine zeitabhängige Wellenfunktion $\psi(x,t)$.
\item[Lösung:]
Durch scharfes hinsehen wieht man:
\[\psi(x)=\frac{\sqrt 3}{2}\psi_1(x)+\frac{1}{2}\psi_2(x)\]
Der Faktor $\left(\frac{m\omega}{\pi\hbar}\right)^{1/4}e^{-\frac{m\omega}{2\hbar}x^2}$ kommt in allen Eigenfunktionen vor. Betrachte also nur die Klammer, $\psi_1$ ist die einzige Funktion die $x^1$ enthält und $\psi_2$ die einzige mit $x^2$.\\
Es gilt
\[\psi(x,t)=\sum_k N_k e^{-iE_k t/\hbar}\ket{k}=\sum_k N_k e^{-iE_k t /\hbar}\psi_k(x)\]
Es ist also
\begin{align*}
\psi(x,t)	&=N_0 e^{-iE_0 t/\hbar}\psi_0(x) + N_1 e^{-iE_1 t/\hbar}\psi_1(x) + N_2 e^{-iE_2 t/\hbar}\psi_2(x) \\
&= \frac{\sqrt 3}{2} e^{-\frac{3}{2}i\omega t}\psi_1(x) + \frac{1}{2} e^{-\frac{5}{2}i\omega t}\psi_2(x) \\
\end{align*}

\item Berechnen Sie Orts- und Impulsmittelwert $\braket{x}=\braket{\psi|x|\psi}$ und $\braket{p}=\braket{\psi|p|\psi}$ im Zustand $\psi(x,t)$
\item[Lösung:]
\begin{align*}
\braket{x}	&=\braket{\psi|x|\psi} = \frac 12 \sqrt{\frac{2\hbar}{m\omega}} \psi^\ast (\hat a + \hat a^\dagger) \psi     \frac 12 \sqrt{\frac{2\hbar}{m\omega}}(N_0^\ast \bra 0 + N_1^\ast \bra 1 + N_2^\ast \bra 2)\:(\hat a +\hat a^\dagger)\:(N_0 \ket 0 + N_1 \ket 1 + N_2 \ket 2) \\
\intertext{Da die $N_i$ rein reell sind, ist $N_i=N_i^\ast$. Außerdem ist $N_0=0$}
			&= \frac 12 \sqrt{\frac{2\hbar}{m\omega}}\big(N_1 \bra 1 + N_2 \bra 2\big)(\hat a +\hat a^\dagger) \big(N_1 \ket 1 + N_2 \ket 2\big) \\
			&= \frac 12 \sqrt{\frac{2\hbar}{m\omega}}\big(N_1 \bra 1 + N_2 \bra 2\big)\:\hat a\big(N_1 \ket 1 + N_2 \ket 2\big) \\
			&+ \frac 12 \sqrt{\frac{2\hbar}{m\omega}}\big(N_1 \bra 1 + N_2 \bra 2\big)\:\hat a^\dagger\big(N_1 \ket 1 + N_2 \ket 2\big) \\
			&= \frac 12 \sqrt{\frac{2\hbar}{m\omega}}\big(N_1 \bra 1 + N_2 \bra 2\big)\big(N_1 \ket 0 + \sqrt 2 N_2  \ket 1\big) \\
			&+ \frac 12 \sqrt{\frac{2\hbar}{m\omega}}\big(N_1 \bra 1 + N_2 \bra 2\big)\big(\sqrt 2 N_1 \ket 2 + \sqrt 3 N_2 \ket 3\big) \\
			&= \frac 12 \sqrt{\frac{2\hbar}{m\omega}} \Big( N_1 N_1\braket{1|0} + \sqrt 2 N_1 N_2 \braket{1|1} + N_2 N_0 \braket{2|0} + \sqrt 2 N_2 N_2 \braket{2|1} \\
			&\qquad + \sqrt 2 N_1 N_1 \braket{1|2} + \sqrt 3 N_1 N_2 \braket{1|3} + \sqrt 2 N_2 N_1 \braket{2|2} + \sqrt 3 N_2 N_2 \braket{2|3}\Big)			
			\intertext{Wegen $\braket{i|j}=\delta_{i,j}$ vereinfacht sich der Term zu}
			&= \frac 12 \sqrt{\frac{2\hbar}{m\omega}}\Big(\sqrt 2 N_1 N_2 + \sqrt 2 N_2 N_1  \Big)	\\
			\intertext{$N_1=\frac{\sqrt 3}{2}, N_2=\frac{1}{2}$}
			&= \sqrt{\frac{2\hbar}{m\omega}}\Big(\sqrt 2 \frac{\sqrt 3}{2} \frac{1}{2}  \Big)	\\
			&= \sqrt{\frac{3\hbar}{2m\omega}}	\\
			&\Big(= \int_{-\infty}^{\infty} x |\psi(x)|^2\:dx\Big)
\end{align*}
Da $\braket{x}$ nicht von $t$ abhängt, ist
\begin{align*}
\braket{p}	&= \frac{d\braket{x}}{dt} \\
			&= 0
\end{align*}

\item Überprüfen Sie Heisenbergs Prinzip
\[(\Delta x)^2(\Delta p)^2\geq\frac{\hbar^2}{4}\]
\item[Lösung:]
Zunächst bemerken wir, dass
\begin{align*}
\hat x &= \frac 12 \sqrt{\frac{2\hbar}{m\omega}} (\hat a + \hat a^\dagger)		\\
\Rightarrow \hat x^2 	&= \frac 12 \frac{\hbar}{m\omega}(\hat a + \hat a^\dagger)^2 \\
\hat{p}	&= \frac{\sqrt {2\hbar m \omega}}{2i} (\hat a - \hat a^\dagger) \\
\Rightarrow \hat p^2
			&= -\frac{\hbar m \omega}{2} (\hat a - \hat a^\dagger)^2
\end{align*}
Berechne dann $\braket{x^2}$:
\begin{align*}
\braket{x^2}	&= \braket{\psi|\hat x^2|\psi}		\\
				&= \frac 12 \frac{\hbar}{m\omega} \psi^\ast (\hat a + \hat a^\dagger)^2 \psi	\\
				&= \frac 12 \frac{\hbar}{m\omega} 
				(N_0^\ast\bra 0 + N_1^\ast \bra 1 + N_2^\ast \bra 2)
				\:(\hat a\hat a + \hat a^\dagger \hat a + \hat a \hat a^\dagger + \hat a^\dagger\hat a^\dagger)\:
				(N_0\ket 0 + N_1 \ket 1 + N_2 \ket 2)
\intertext{Es ist wieder $N_0^\ast=N_0=0$ und $N_1^\ast = N_1, N_2^\ast = N_2$}
				&= \frac 12 \frac{\hbar}{m\omega}
				(N_1 \bra 1 + N_2 \bra 2)
				\:(\hat a\hat a + \hat a^\dagger \hat a + \hat a \hat a^\dagger + \hat a^\dagger\hat a^\dagger)\:
				(N_1 \ket 1 + N_2 \ket 2)		\\
				&=  \frac 12 \frac{\hbar}{m\omega} \Big(	
				(N_1 \bra 1 + N_2 \bra 2)\:\hat a\hat a(N_1\ket 1 +N_2\ket 2) \\
				&\phantom{=\frac 12 \frac{\hbar}{m\omega} \Big(}
					 + (N_1 \bra 1 + N_2 \bra 2)\:\hat a\hat a^\dagger(N_1\ket 1 +N_2\ket 2) \\
				&\phantom{=\frac 12 \frac{\hbar}{m\omega} \Big(}
					+ (N_1 \bra 1 + N_2 \bra 2)\:\hat a^\dagger\hat a(N_1\ket 1 +N_2\ket 2) \\
				&\phantom{=\frac 12 \frac{\hbar}{m\omega} \Big(}
					+ (N_1 \bra 1 + N_2 \bra 2)\:\hat a^\dagger\hat a^\dagger(N_1\ket 1 +N_2\ket 2) \Big)\\
				%
				&=  \frac 12 \frac{\hbar}{m\omega} \Big(
					(N_1 \bra 1 + N_2 \bra 2)\:\hat a(N_1\ket 0 + \sqrt 2 N_2\ket 1) \\
				&\phantom{=\frac 12 \frac{\hbar}{m\omega} \Big(}
					+ (N_1 \bra 1 + N_2 \bra 2)\:\hat a(\sqrt 2 N_1\ket 2 +\sqrt 3 N_2\ket 3) \\
				&\phantom{=\frac 12 \frac{\hbar}{m\omega} \Big(}
					+ (N_1 \bra 1 + N_2 \bra 2)\:\hat a^\dagger(N_1\ket 0 +\sqrt 2 N_2\ket 1) \\
				&\phantom{=\frac 12 \frac{\hbar}{m\omega} \Big(}
					+ (N_1 \bra 1 + N_2 \bra 2)\:\hat a^\dagger(\sqrt 2 N_1\ket 2 + \sqrt 3 N_2\ket 3) \Big)\\
				%
				&=  \frac 12 \frac{\hbar}{m\omega} \Big(
					(N_1 \bra 1 + N_2 \bra 2)\:(N_1\cdot 0  + \sqrt 2 N_2\ket 0) \\
				&\phantom{=\frac 12 \frac{\hbar}{m\omega} \Big(}
					+ (N_1 \bra 1 + N_2 \bra 2)\:(\sqrt 2 \sqrt 2 N_1\ket 1 + \sqrt 3 \sqrt 3 N_2\ket 2) \\
				&\phantom{=\frac 12 \frac{\hbar}{m\omega} \Big(}
					+ (N_1 \bra 1 + N_2 \bra 2)\:(N_1\ket 1 +\sqrt 2 \sqrt 2  N_2\ket 2) \\
				&\phantom{=\frac 12 \frac{\hbar}{m\omega} \Big(}
					+ (N_1 \bra 1 + N_2 \bra 2)\:(\sqrt 2 \sqrt 3 N_1\ket 3 + \sqrt 3 \sqrt 4 N_2\ket 4) \Big) \\
				%
				&=  \frac 12 \frac{\hbar}{m\omega} \Big( 0 + (2 N_1 N_1 + 3 N_2 N_2)  + (N_1 N_1 + 2 N_2 N_2) + 0\Big) \\
				&=  \frac 12 \frac{\hbar}{m\omega} \Big( 3 N_1^2 + 5 N_2^2\Big) \\
				&=  \frac 12 \frac{\hbar}{m\omega} \Big( 3 \frac 34 + 5 \frac 14 \Big) \\
				&=  \frac 74 \frac{\hbar}{m\omega}\\
\intertext{Integration von $\int_{-\infty}^{\infty} \psi^\ast(x) \hat x^2 \psi(x)\:dx$ ergibt das Gleiche (mit CAS)}
\end{align*}
Bei der Lösung ist es enorm hilfreich, dass $\braket{i|j} = \delta_{i,j}$, d.h. fast alle Brakets fallen weg. Man kann sich außerdem überlegen, wie $\hat a \hat a, \hat a\hat a^\dagger, \dots$ aussehen und spart sich damit einen Zwischenschritt und viel Schreibarbeit.

\begin{align*}
\braket{p^2}	&= \braket{\psi|\hat p^2|\psi}		\\
				&= -\frac{\hbar m \omega}{2} \psi^\ast(\hat a - \hat a^\dagger)^2\psi	\\
				&= -\frac{\hbar m \omega}{2}
				(N_0^\ast\bra 0 + N_1^\ast \bra 1 + N_2^\ast \bra 2)
				\:(\hat a\hat a - \hat a^\dagger \hat a - \hat a \hat a^\dagger + \hat a^\dagger\hat a^\dagger)\:
				(N_0\ket 0 + N_1 \ket 1 + N_2 \ket 2)
\intertext{Es ist wieder $N_0^\ast=N_0=0$ und $N_1^\ast = N_1, N_2^\ast = N_2$}
				&= -\frac{\hbar m \omega}{2}
				(N_1 \bra 1 + N_2 \bra 2)
				\:(\hat a\hat a - \hat a^\dagger \hat a - \hat a \hat a^\dagger + \hat a^\dagger\hat a^\dagger)\:
				(N_1 \ket 1 + N_2 \ket 2)		\\
				&= -\frac{\hbar m \omega}{2}
				(N_1 \bra 1 + N_2 \bra 2)
				\:(\hat a\hat a + \hat a^\dagger \hat a + \hat a \hat a^\dagger + \hat a^\dagger\hat a^\dagger)\:
				(N_1 \ket 1 + N_2 \ket 2)		\\
				&=  -\frac{\hbar m \omega}{2}  \Big(
					(N_1 \bra 1 + N_2 \bra 2)\:\hat a\hat a(N_1\ket 1 +N_2\ket 2) \\
				&\phantom{=\frac 12 \frac{\hbar}{m\omega} \Big(}
					- (N_1 \bra 1 + N_2 \bra 2)\:\hat a\hat a^\dagger(N_1\ket 1 -N_2\ket 2) \\
				&\phantom{=\frac 12 \frac{\hbar}{m\omega} \Big(}
					- (N_1 \bra 1 + N_2 \bra 2)\:\hat a^\dagger\hat a(N_1\ket 1 -N_2\ket 2) \\
				&\phantom{=\frac 12 \frac{\hbar}{m\omega} \Big(}
					+ (N_1 \bra 1 + N_2 \bra 2)\:\hat a^\dagger\hat a^\dagger(N_1\ket 1 +N_2\ket 2) \Big)\\
				%
				&=  -\frac{\hbar m \omega}{2}  \Big(
					(N_1 \bra 1 + N_2 \bra 2)\:\hat a (N_1\ket 0 + \sqrt 2 N_2\ket 1) \\
				&\phantom{=\frac 12 \frac{\hbar}{m\omega} \Big(}
					- (N_1 \bra 1 + N_2 \bra 2)\:\hat a (\sqrt 2 N_1\ket 2 - \sqrt 3 N_2\ket 3) \\
				&\phantom{=\frac 12 \frac{\hbar}{m\omega} \Big(}
					- (N_1 \bra 1 + N_2 \bra 2)\:\hat a^\dagger (N_1\ket 0 - \sqrt 2 N_2\ket 1) \\
				&\phantom{=\frac 12 \frac{\hbar}{m\omega} \Big(}
					+ (N_1 \bra 1 + N_2 \bra 2)\:\hat a^\dagger (\sqrt 2 N_1\ket 2 + \sqrt  3 N_2\ket 3) \Big)\\	
				%
				&=  -\frac{\hbar m \omega}{2}  \Big(
					(N_1 \bra 1 + N_2 \bra 2)\:(0 + \sqrt 2 N_2\ket 0) \\
				&\phantom{=\frac 12 \frac{\hbar}{m\omega} \Big(}
					- (N_1 \bra 1 + N_2 \bra 2)\: (\sqrt 2 \sqrt 2 N_1\ket 1 - \sqrt 3 \sqrt 3  N_2\ket 2) \\
				&\phantom{=\frac 12 \frac{\hbar}{m\omega} \Big(}
					- (N_1 \bra 1 + N_2 \bra 2)\: (N_1\ket 1 - \sqrt 2 \sqrt 2 N_2\ket 2) \\
				&\phantom{=\frac 12 \frac{\hbar}{m\omega} \Big(}
					+ (N_1 \bra 1 + N_2 \bra 2)\:(\sqrt 2 \sqrt 3 N_1\ket 3 + \sqrt  3 \sqrt 4 N_2\ket 4) \Big)\\		
				%
				&=  -\frac{\hbar m \omega}{2}  \Big(
					0 -(2N_1^2 + 3 N_2^2) - (N_1^2 + 2 N_2^2) + 0
				\Big)\\	
				&= \frac{\hbar m \omega}{2}  \Big( 3N_1^2 + 5 N_2^2 \Big)\\
				&= \frac{\hbar m \omega}{2} \Big( 3\frac{3}{4} + 5 \frac{1}{4} \Big) \\
				&= \frac 74 \hbar m \omega \\
\end{align*}

Damit ist 
\begin{align*}
\Delta x	&= \sqrt{\braket{x^2}-\braket{x}^2}	\\
			&= \sqrt{\frac 74 \frac{\hbar}{m\omega} - \frac{3\hbar}{2m\omega}} \\
			&= \frac{1}{2}\sqrt{\frac{\hbar}{m\omega}} \\
\Delta p	&= \sqrt{\braket{p^2}-\braket{p}^2}	\\
			&= \sqrt{\braket{p^2}-0}	\\
			&= \sqrt{\frac 74 \hbar m \omega}
\end{align*}
\[(\Delta x)^2 (\Delta p)^2 = \frac{7}{16}\hbar^2 \geq \frac{4}{16}\hbar^2 = \frac{\hbar^2}{4}\]
\end{enumerate}

\section{Teilchen mit Spin im magnetischen Feld}
Betrachten Sie ein Elektron in einem Überlagerungszustand durch
\[\psi=\frac{1}{\sqrt 2}\begin{pmatrix}\frac{1+\sqrt 3}{2}\\\frac{1-\sqrt 3}{2}\end{pmatrix}\]

\begin{enumerate}[a)]
\item Bestimmen Sie die Wahrscheinlichkeit im Zustand $\psi$ Spin $\frac{\hbar}{2}$ in $x$- bzw. $y$-Richtung zu messen.
\item[Lösung:]
Wahrscheinlichkeit im Zustand $\psi$ den Spin $\frac\hbar 2$ in $x$-Richtung zu messen:
\begin{align*}
\braket{\psi|S_x|\psi}	
	&= 
	\left(\frac 1{\sqrt 2} \begin{pmatrix}\frac{1+\sqrt 3}{2} & \frac{1-\sqrt 3}{2}\end{pmatrix}\right)^\ast
	\frac \hbar 2 \begin{pmatrix} 0&1\\1&0 \end{pmatrix}
	\frac 1{\sqrt 2}\begin{pmatrix}\frac{1+\sqrt 3}{2}\\\frac{1-\sqrt 3}{2}\end{pmatrix}\\
	&= 	
	\frac \hbar 4 \begin{pmatrix}\frac{1+\sqrt 3}{2} & \frac{1-\sqrt 3}{2}\end{pmatrix}
	\begin{pmatrix} 0&1\\1&0 \end{pmatrix}
	\begin{pmatrix}\frac{1+\sqrt 3}{2}\\\frac{1-\sqrt 3}{2}\end{pmatrix}\\
	&=
	\frac \hbar {16} \begin{pmatrix}1+\sqrt 3 & 1-\sqrt 3\end{pmatrix}
	\begin{pmatrix}1-\sqrt 3\\1+\sqrt 3\end{pmatrix}\\
	&=
	\frac \hbar {16} 2 (1 -\sqrt 3 +\sqrt 3 -\sqrt 3 \sqrt 3)\\ 
	&=
	\frac \hbar {16} 2 (1 - 3) = -\frac \hbar 2
\end{align*}
Genauso für die $y$-Richtung:
\begin{align*}
\braket{\psi|S_y|\psi}	
	&= 
	\left(\frac 1{\sqrt 2} \begin{pmatrix}\frac{1+\sqrt 3}{2} & \frac{1-\sqrt 3}{2}\end{pmatrix}\right)^\ast
	\frac \hbar 2 \begin{pmatrix} 0&-i\\i&0 \end{pmatrix}
	\frac 1{\sqrt 2}\begin{pmatrix}\frac{1+\sqrt 3}{2}\\\frac{1-\sqrt 3}{2}\end{pmatrix}\\
	&= 	
	\frac \hbar 4 \begin{pmatrix}\frac{1+\sqrt 3}{2} & \frac{1-\sqrt 3}{2}\end{pmatrix}
	\begin{pmatrix} 0&-i\\i&0 \end{pmatrix}
	\begin{pmatrix}\frac{1+\sqrt 3}{2}\\\frac{1-\sqrt 3}{2}\end{pmatrix}\\
	&=
	\frac \hbar {16} \begin{pmatrix}1+\sqrt 3 & 1-\sqrt 3\end{pmatrix}
	\begin{pmatrix}-i+i\sqrt 3\\i+i\sqrt 3\end{pmatrix}\\
	&=
	\frac {i\hbar}{16} \begin{pmatrix}1+\sqrt 3 & 1-\sqrt 3\end{pmatrix}
	\begin{pmatrix}-1+\sqrt 3\\1+\sqrt 3\end{pmatrix}\\
	&=
	\frac {i\hbar}{16} (-1 + \sqrt 3 -\sqrt 3 + 3 + 1 + \sqrt 3  -\sqrt 3 -3) \\
	&=
	\frac{i\hbar}{16} 0 = 0
\end{align*}

\item Das Teilchen ruht in einem homogenen magnetischen Feld $\vec B = B_0 e_z$. Der Hamiltonoperator lautet
\[\hat H = -\vec \mu \vec B = \frac em \vec S \vec B \]
Berechnen Sie die Eigenvektoren und die Eigenwerte (Energien $E_1, E_2$) von $\hat H$. Ist $\psi$ ein stationärer Zustand? Begründen Sie!
\item[Lösung:]
\begin{align*}
\hat H &= \frac em \vec S \vec B = \frac em \vec S B_0 e_x = \frac em B_0 S_x = \frac em B_0 \frac \hbar 2 \begin{pmatrix}0&1\\1&0\end{pmatrix} \\
&= \frac{e\hbar B_0}{2m}\Sx
\end{align*}

Eigenwerte $E_i$ von $\hat  H$ erfüllen
\begin{align*}
\hat H \psi_n						&= E_i \psi_n			\\
\frac{e\hbar B_0}{2m}\Sx \psi_n
&= 
E_i \psi_n
\end{align*}

$\sigma_x = \begin{pmatrix}0&1\\1&0\end{pmatrix}$ hat die Eigenwerte $\lambda_1 = 1, \lambda_2 = -1$, also ist
\begin{align*}
E_1 =  \frac{e\hbar B_0}{2m}, \qquad& \psi_1 \in \left[\vecz 11 \right]\\
E_2 = -\frac{e\hbar B_0}{2m}, \qquad& \psi_2 \in \left[\vecz {-1}1 \right]
\end{align*}
Offensichtlich ist $\psi$ weder im Eigenraum $\left[\vecz 11 \right]$ noch im Eigenraum $\left[\vecz {-1}1 \right]$ enthalten deshalb kein Eigenvektor und nicht stationär.


\item Bestimmen Sie nun die Wahrscheinlichkeit im Zustand $\psi$ die Energie $E_1$ bzw. $E_2$ zu messen. Geben Sie anschließend den Mittelwert der Energien
\[\braket{E} = \braket{\psi|\hat H|\psi}\]
an.
\item[Lösung:]Wahrscheinlichkeit im beliebigen Zustand $\phi$ die Energie E zu messen ist
\[P_E = |\braket{v|\phi}|^2\], wobei v der Eigenvektor zum Eigenwert E ist.

\begin{align*}
E_1 &=  \frac{e\hbar B_0}{2m}, \quad v_1 = \frac{1}{\sqrt 2} \left[\vecz 11 \right]\\
E_2 &= -\frac{e\hbar B_0}{2m}, \quad v_2 = \frac{1}{\sqrt{2}}\left[\vecz 1{-1} \right]\\
P_1 &= |\braket{v_1|E_1}|^2 = \frac{1}{4}, \quad P_2 = |\braket{v_2|E_2}|^2 = \frac{3}{16}\\
\hat H &= \frac{e\hbar B_0}{2m}\Sx \\
\braket{E} &= \braket{\psi|\hat H|\psi} = \braket{\psi|\frac{e\hbar B_0}{2m}\Sx|\psi} = \frac{e\hbar B_0}{2m}\braket{\psi|\Sx|\psi} = -\frac{e\hbar B_0}{2m}\frac \hbar 2 = -\frac{e\hbar^2 B_0}{4m}
\end{align*}

\item Bestimmen Sie die Zeitentwicklung des Zustandes $\psi(t)$ mit Hilfe der Zerlegung von $\psi$ in die Eigenzustände von $\hat H$.

\item[Lösung:]
Die Eigenzustände von $\hat H$ sind die Eigenvektoren der Paulimatrix $\sigma_x$. Eigenwerte $E_1$ und $E_2$ sind bereits bekannt.

\begin{align*}
\Psi(t) = N_1e^\frac{-iE_1t}{\hbar}v_1 + N_2e^\frac{-iE_2t}{\hbar}v_2\\
t = 0 \Rightarrow \Psi(0) = N_1v_1 + N_2v_2 = \frac{1}{2 \sqrt{2}} \vecz {1+\sqrt{3}}{1-\sqrt{3}}
\end{align*}
Durch scharfes hinsehen erkennt man
\begin{align*}
N_1 = \frac{1}{2}, \quad N_2 = \frac{1}{2}\sqrt{3}
\end{align*}

\end{enumerate}


\end{document}
