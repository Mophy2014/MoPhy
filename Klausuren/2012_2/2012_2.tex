\documentclass{scrartcl}

\usepackage{amsmath}
\usepackage{amssymb}
\usepackage{enumerate}
\usepackage[utf8]{inputenc}
\usepackage{braket}

\begin{document}

\section{Harmonischer Oszillator}

Gegeben sei ein quantenmechanischer harmonischer Oszillator beschrieben durch die eine Wellenfunktion
\[\psi(x,0)=\left(\frac{m\omega}{\pi \hbar}\right)^{1/4}\left(\frac{m\omega}{\sqrt 2 \hbar}x^2+\sqrt{\frac{3m\omega}{2\hbar}}x-\frac{1}{2\sqrt 2}\right)e^{-\frac{m\omega}{2\hbar}x^2}\]
Und den Eigenfunktionen
\begin{align*}
\psi_0(x)	&=\left(\frac{m\omega}{\pi\hbar}\right)^{1/4}e^{-\frac{m\omega}{2\hbar}x^2}	\\
\psi_1(x)	&=\left(\frac{m\omega}{\pi\hbar}\right)^{1/4}\sqrt{\frac{2m\omega}{\hbar}}xe^{-\frac{m\omega}{2\hbar}x^2}	\\
\psi_2(x)	&=\left(\frac{m\omega}{\pi\hbar}\right)^{1/4}\frac{1}{\sqrt 2}\left(\frac{2m\omega}{\hbar}x^2-1\right)e^{-\frac{m\omega}{2\hbar}x^2}
\end{align*}
und der Energie 
\[E_n =\left(n+\frac{1}{2}\right)\hbar\omega\]

\begin{enumerate}[a)]
\item Schreiben Sie die Wellenfunktion als eine Summe von Eigenfunktionen des harmonischen Oszillators und bestimmen Sie eine zeitabhängige Wellenfunktion $\psi(x,t)$.
\item[Lösung:]
Durch scharfes hinsehen wieht man:
\[\psi(x)=\frac{\sqrt 3}{2}\psi_1(x)+\frac{1}{2}\psi_2(x)\]
Der Faktor $\left(\frac{m\omega}{\pi\hbar}\right)^{1/4}e^{-\frac{m\omega}{2\hbar}x^2}$ kommt in allen Eigenfunktionen vor. Betrachte also nur die Klammer, $\psi_1$ ist die einzige Funktion die $x^1$ enthält und $\psi_2$ die einzige mit $x^2$.\\
Es gilt
\[\psi(x,t)=\sum_k N_k e^{-iE_k t/\hbar}\ket{k}=\sum_k N_k e^{-iE_k t /\hbar}\psi_k(x)\]
Es ist also
\begin{align*}
\psi(x,t)	&=N_0 e^{-iE_0 t/\hbar}\psi_0(x) + N_1 e^{-iE_1 t/\hbar}\psi_1(x) + N_2 e^{-iE_2 t/\hbar}\psi_2(x) \\
&= \frac{\sqrt 3}{2} e^{-\frac{3}{2}i\omega t}\psi_1(x) + \frac{1}{2} e^{-\frac{5}{2}i\omega t}\psi_2(x) \\
\end{align*}

\item Berechnen Sie Orts- und Impulsmittelwert $\braket{x}=\braket{\psi|x|\psi}$ und $\braket{p}=\braket{\psi|p|\psi}$ im Zustand $\psi(x,t)$
\item[Lösung:]
\begin{align*}
\braket{x}&=\int_{-\infty}^{\infty} x |\psi(x)|^2\:dx = 0  \qquad\text{(ungerade Funktion)} \\
\intertext{Um $\braket{p}$ zu berechnen, kann man entweder $\psi(x)$ in Impulsdarstellung bringen (Fouriertransformation) oder man benutzt den Operator $\hat p=-i\hbar \frac{\partial}{\partial x}$}
\braket{p}&=\int_{-\infty}^{\infty} \psi^\ast(x) \hat p \psi(x)\: dx=-i\hbar\int_{-\infty}^{\infty}\psi^\ast(x)\frac{\partial}{\partial x}\psi(x)\:dx
\intertext{Dieser Lösungsweg ist scheinbar nicht für die Klausur geeignet, da es augenscheinlich sehr aufwendig ist, $\psi(x)$ nach $x$ abzuleiten. Die Multiplikation mit $\psi^\ast(x)$ und anschließende Integration sind mindestens ebenso aufwendig. Stattdessen:}
\frac{d \braket{x}}{dt}=0 \Rightarrow \braket{p}=0
\end{align*}

\item Überprüfen Sie Heisenbergs Prinzip
\[(\Delta x)^2(\Delta p)^2\geq\frac{\hbar^2}{4}\]
\item[Lösung:]
Berechne zuerst $\braket{x^2}$ und $\braket{p^2}$:
\begin{align*}
\braket{x^2}	&= \int_{-\infty}^{\infty} x^2 |\psi(x)|^2\:dx = 2\int_{0}^{\infty}x^2
\end{align*}

\end{enumerate}


\end{document}