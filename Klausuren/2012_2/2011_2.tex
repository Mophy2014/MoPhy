\documentclass{scrartcl}

\usepackage{amsmath}
\usepackage{amssymb}
\usepackage{enumerate}
\usepackage[utf8]{inputenc}
\usepackage{braket}
\usepackage{tikz}
\usepackage[ngerman]{babel}

\newcommand{\Sx}{\begin{pmatrix}0&1\\1&0\end{pmatrix}}
\newcommand{\vecz}[2]{\begin{pmatrix}#1 \\ #2\end{pmatrix}}

\allowdisplaybreaks % allow breaking align enviornment over multiple pages

\begin{document}

\section{Teilchen mit Spin im magnetischen Feld}

Betrachten Sie ein Elektron mit einem Magnetischen Moment $\vec \mu$
\[\vec \mu = -\frac{e}{mc}\vec S=-\frac{e}{mc}\frac{\hbar }{2}\sigma \qquad(e>0)\]

Das Teilchen ruht in einem homogenen magnetischen Feld $\vec B=B_0 (\vec e_x + \sqrt 3 \vec e_y)$. Am Anfang bei $t=0$ befindet sich das Elektron in einem Zustand, in dem die Wahrscheinlichkeit einen Wert $\frac \hbar 2$ für die $z$-Komponente des Spins $\hat S_z$ zu messen $P_z\left(\frac \hbar 2\right)=\frac 12$ ist, d.h. $P_z\left(-\frac \hbar 2\right)=\frac 12$.
\begin{enumerate}[a)]
\item Schreiben Sie den Anfangszustand $\psi$ als Überlagerung von Eigenzuständen der Paulimatrix $\sigma_z$. Zur Erinnerung: Die Eigenwerte und Eigenvektoren von $\sigma_z$ lauten:
\[\lambda_1 = 1, v_1 = \vecz 10 \qquad \lambda_2 = -1, v_2 = \vecz 01\]

\item[Lösung:]
Gesucht ist die Wahrscheinlichkeit, $\frac \hbar 2 $ in $z$-Richtung zu messen, berechne deshalb die Eigenwerte und Eigenvektoren von $\sigma_z$ (sind bereits gegeben!). Die Eigenvektoren repräsentieren jeweils den Zustand \glqq Spin $\pm\frac{\hbar}{2}$\grqq\ in $z$-Richtung. $\psi$ ist die normierte Superposition der Eigenvektoren (=Zustände) gewichtet mit den Wahrscheinlichkeiten
\begin{align*}
\psi' 	&= \frac 12 v_1 + \frac 12 v_2	\\
		&= \frac 12 \vecz 11 \\
\intertext{Normiere $\psi'$ zu $\psi$:}
\psi  &= \frac {\psi'}{\|\psi'\|} = \frac 1{\sqrt 2} \vecz 11
\end{align*}

\item Der Hamiltonoperator des Systems lautet $\hat H = - \vec \mu \cdot \vec B$. Berechnen Sie die Eigenvektoren und die Eigenwerte (die Energien $E_1, E_2$) von $\hat H$. Ist $\psi$ ein stationärer Zustand? Begründen Sie.
\item[Lösung:]
\begin{align*}
\hat H  &= -\vec \mu \cdot \vec B \\
	&= -\frac{e}{mc} \frac \hbar 2 \sigma B_0(\vec e_x +\sqrt 3 \vec e_z) \\
	&= -\frac{e\hbar B_0}{2mc}\begin{pmatrix}
	\sigma_x\\\sigma_y\\\sigma_z
	\end{pmatrix}
	\left(\begin{pmatrix}1\\0\\0\end{pmatrix} + \sqrt 3 \begin{pmatrix}0\\0\\1\end{pmatrix}\right) \\
 	&=  -\frac{e\hbar B_0}{2 mc}(\sigma_x + \sqrt 3 \sigma_z)\\
 	&= \underbrace{-\frac{e\hbar B_0}{2 mc}}_{=:\Lambda}\begin{pmatrix}
 	\sqrt 3 & 1 \\ 1 & -\sqrt 3
 	\end{pmatrix}
\end{align*}
Eigenwerte von $\hat H$:
\begin{align*}
\left|\begin{matrix}
\sqrt 3 -E & \Lambda  \\ \Lambda & -\sqrt 3-E
\end{matrix}\right| &= \
\end{align*}
Eigenvektoren von $\hat H$:
\begin{itemize}
\item [$\lambda_1 = 2\Lambda$:]
\begin{align*}
\begin{pmatrix}
-2\Lambda & (1-i\sqrt 3)\Lambda \\(1+i\sqrt 3)\Lambda & -2\Lambda
\end{pmatrix} \leadsto
\begin{pmatrix}
-2 & 1-i\sqrt 3 \\
2 & \frac{-2 \cdot 2}{1+i\sqrt 3}
\end{pmatrix} \\
\leadsto\begin{pmatrix}
-2 & 1-i\sqrt 3 \\
0 & \frac{(1-i\sqrt 3)(1+i\sqrt 3)}{(1+i\sqrt 3)}-\frac{4}{1+i\sqrt 3}
\end{pmatrix}
\leadsto\begin{pmatrix}
1 & -\frac 12 + \frac{i\sqrt 3}2 \\
0 & \frac{4}{(1+i\sqrt 3)}-\frac{4}{1+i\sqrt 3}
\end{pmatrix}\\
\leadsto\begin{pmatrix}
1 & -\frac 12 + \frac{i\sqrt 3}2 \\
0 & 0
\end{pmatrix}
\end{align*}
$\Rightarrow -1$-Trick: Eigenraum zu $\lambda_1=2$ ist $\left[\vecz{\frac 12 -\frac{i\sqrt 3}2}1 \right]$. Wähle $v_1=\vecz{\frac 12 -\frac{i\sqrt 3}2}1$.

\item[$\lambda_2=-2\Lambda$:]
\begin{align*}
\begin{pmatrix}
2\Lambda & (1-i\sqrt 3)\Lambda \\ (1+i\sqrt 3)\Lambda & 2\Lambda
\end{pmatrix} \leadsto
\begin{pmatrix}
2 & 1-i\sqrt 3 \\ -2 & \frac{-2 \cdot 2}{1+i\sqrt 3}
\end{pmatrix} \\
\leadsto
\begin{pmatrix}
2 & 1-i\sqrt 3 \\ 0 & \frac{-4}{1+i\sqrt 3} +\frac{(1-i\sqrt 3) (1+i\sqrt 3)}{1+i\sqrt 3}
\end{pmatrix}
\leadsto
\begin{pmatrix}
1 & \frac 12 - \frac{i\sqrt 3}2 \\ 0 & 0
\end{pmatrix}
\end{align*}
$\Rightarrow -1$-Trick: Eigenraum zu $\lambda_2=-2$ ist $\left[\vecz{-\frac 12 +\frac{i\sqrt 3}2}1 \right]$. Wähle $v_2=\vecz{-\frac 12 +\frac{i\sqrt 3}2}1$.
\end{itemize}

$\psi$ ist nicht stationär, da $\psi$ nicht das vielfache eines Eigenvektors ist.
\item Schreiben Sie $\psi$ als Linearkombination der Eigenzustände von $\hat H$ um.
\item[Lösung:]
\begin{align*}
\psi = \frac 1{\sqrt 2} \vecz 11 &= \alpha v_1 + \beta v_2 \\
	&= \alpha \vecz{\frac 12 -\frac{i\sqrt 3}2}1 + \beta \vecz{-\frac 12 +\frac{i\sqrt 3}2}1 \\
	&= \begin{pmatrix} 
	\frac 12 -\frac{i\sqrt 3}2 & -\frac 12 +\frac{i\sqrt 3}2 \\
	1 & 1 	\end{pmatrix} \vecz \alpha \beta
\end{align*}
Durch scharfes Hinsehen erkennt man (nicht wirklich...):
\[\alpha = \frac{3+\sqrt{3} i}{4 \sqrt{2}}, \qquad 
   \beta = \frac{1-\sqrt{3} i}{4 \sqrt{2}} \]

\item Bestimmen Sie nun die Wahrscheinlichkeit in dem Zustand $\psi$ die Energie $E_1$ und $E_2$ zu messen. Geben Sie anschließend den Mittelwert der Energie
\[\braket{E} = \psi^+ H \psi\]

\item[Lösung:]
Wahrscheinlichkeit im Zustand $\psi$ Energie $E_n$ zu messen ist 





\end{enumerate}


\end{document}