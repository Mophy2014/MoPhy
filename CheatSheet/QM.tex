\documentclass[landscape,8pt]{scrartcl}

\usepackage[left=1cm,top=1cm,right=1cm,bottom=1cm]{geometry}
\usepackage{multicol}
\usepackage{braket}
\usepackage{amsmath}
\usepackage{amssymb}
\usepackage[utf8]{inputenc}




\begin{document}
\begin{multicols}{3}

\noindent Zeitabhängige Wellenfunktion:
\[
	\psi(x,t)
		= \alpha e^{-itE_0/\hbar}\ket 0 
		+ \beta  e^{-itE_1/\hbar}\ket 1  
		+ \cdots
\]
Mit den Energien $E_n = \left(n+\frac 12\right) \omega \hbar $ \\
Und den Eigenfunktionen $\ket n = \psi_n(x)$\\

\subsection{Operatoren}
\begin{tabular}{ll}
Ortsoperator 			& $\hat x = x = i \hbar \frac{\partial}{\partial p}$	\\
Impulsoperator 			& $\hat p = p = -i\hbar \frac{\partial}{\partial x}$	\\
Abstiegsoperator 		& $\hat a = \sqrt{\frac{m \omega}{2\hbar}} \hat x + \frac{i}{\sqrt{2\hbar m \omega}} \hat p$	\\
Aufstiegsoperator 		& $\hat a^\dagger = \sqrt{\frac{m \omega}{2\hbar}} \hat x - \frac{i}{\sqrt{2\hbar m \omega}} \hat p$ \\
Hamiltonoperator		& $\hat H=\frac{\hat p^2}{2m}+\frac{1}{2}m\omega^2\hat x^2 = \frac{\hat p^2}{2m} + V(\hat x)$	\\
Damit gilt				& $\hat x  = \frac 12 \sqrt{\frac{2\hbar}{m\omega}}(\hat a + \hat a^\dagger) $ \\
						& $\hat p  = \frac 1{2i} \sqrt {2\hbar m \omega} (\hat a - \hat a^\dagger) $\\
						& $\hat a\ket n = \sqrt{n}\ket{n-1} \qquad \hat a \ket 0 = 0$ \\
						& $\hat a^\dagger \ket n = \sqrt{n+1}\ket {n+1}$	\\
						& $\ket n = \frac{(\hat a^\dagger)^n}{\sqrt {n!}} \ket 0 $
\end{tabular}

\subsection{Mittelwerte}
\begin{align*}
\braket{x} 	&=\braket{\psi|x|\psi}\\
			&=\int_{-\infty}^{\infty} x\:|\psi(x)|^2\:dx\qquad\text{nur falls } x \text{ kein Operator!}\\
			&=\int_{-\infty}^{\infty} \psi(x)^\ast \hat x\:\psi(x)\:dx		\\
			&=\frac 12 \sqrt{\frac{2\hbar}{m\omega}}(\alpha^\ast \bra 0 + \beta^\ast \bra 1 + \dots) (\hat a + \hat a^\dagger) (\alpha \ket 0 + \beta \ket 1 + \dots) \\
\braket{p}	&=\braket{\psi|x|\psi}	\\
			&=\int_{-\infty}^{\infty} \hat p\:|\psi(x)|^2\:dx= \text{ analog zu}\braket{\psi|x|\psi}\text{.}	\\
\end{align*}


\subsection{Pauli-Matrizen}
\begin{align*}
S_0 &= \frac{\hbar}{2}\sigma_0 = \frac{\hbar}{2} \begin{pmatrix}1&0\\0&1\end{pmatrix}	\\
S_1 &= \frac{\hbar}{2}\sigma_1 = \frac{\hbar}{2} \begin{pmatrix}0&1\\1&0\end{pmatrix}	\\
S_2 &= \frac{\hbar}{2}\sigma_2 = \frac{\hbar}{2} \begin{pmatrix}0&-i\\i&0\end{pmatrix}	\\
S_3 &= \frac{\hbar}{2}\sigma_3 = \frac{\hbar}{2} \begin{pmatrix}1&0\\0&-1\end{pmatrix}	\\
\vec S &= \begin{pmatrix}S_x\\S_y\\S_z\end{pmatrix}			\\
\end{align*}
$S_1 = S_x, \; \sigma_1 = \sigma_x, \; S_2 = S_y \dots$

\begin{align*}
\gamma
E_\text{kin} &= \frac 12 m V^2 = 	\\	
p	&= \sqrt{2 m E_\text{kin}} \\
\end{align*}

\subsection{De Broglie}
\begin{align*}
\lambda = \frac{h}{p} = \frac{h}{m\cdot v}
\end{align*}

\subsection{Relativitätstheorie}
\begin{tabular}{ll}
Lorentzfaktor 	& $\gamma=\frac{1}{\sqrt{1-\frac{v^2}{c^2}}}$ \\
Bewegte Masse	& $m=\gamma m_0$  \qquad \text{ (Mit Ruhemasse $m_0$)} \\
Impuls			& $p=mv = \gamma m_0 v$ \\
Energie			& $E=mc^2=\gamma m_0c^2$ \\
Kinetische Energie 		
				& $E_\text{kin} = \frac 12 m v^2 = \frac \gamma 2 m_0 v^2 = mc^2 - m_0c^2$
\end{tabular} 

\subsection{Konstanten}
\begin{align*}
m_e &= 9.10938291\cdot 10^{-31} \text{ kg} \\
m_p &= 1.67262178\cdot 10^{-27} \text{ kg} \\
m_n &= 1.67492735\cdot 10^{-27} \text{ kg} \\
h &= 6.62606957\cdot 10^{-34} \text{Js} \\
\hbar &= \frac{h}{2\pi} \\
E_\gamma &= h \nu \quad\text{ (Energie des Photons)} \\
p_\gamma &= \frac{E_\gamma}{c} \quad \text{(Impuls des Photons)}\\
\end{align*}

\subsection{Kommutatoren}
\begin{align*}
\left[\hat L_i,\hat x_j\right] &= i\hbar \epsilon_{ijk}\hat x_k	\\
\left[\hat L_i,\hat p_j\right] &= i\hbar \epsilon_{ijk}\hat p_k	\\
\left[\hat A, \hat B \hat C\right]&= \left[ \hat A, \hat B \right] \hat C+ \hat B\left[ \hat A, \hat C \right] \\
\left[\hat A\hat B, \hat C\right]&= \hat A \left[ \hat B, \hat C \right] + \left[ \hat A, \hat C \right] \hat B\\
\end{align*}

\subsection{Drehimpuls}
\begin{align*}
\vec r 	&= \begin{pmatrix}\hat x\\\hat y\\\hat z\end{pmatrix},
\quad \vec p 	= \begin{pmatrix}\hat p_x\\\hat p_y\\\hat p_z\end{pmatrix},
\quad \vec L = \vec p \times \vec r \\
%
\L_x 	&= yp_z - zp_y = -i\hbar \left( y\frac{\partial}{\partial z}-z\frac{\partial}{\partial y}\right) \\
%
\L_y 	&= zp_x - xp_z = -i\hbar \left( z\frac{\partial}{\partial x}-x\frac{\partial}{\partial z}\right) \\
%
\L_z 	&= xp_y - yp_x = -i\hbar \left( x\frac{\partial}{\partial y}-y\frac{\partial}{\partial x}\right)\\
\left[L_i,\hat f\right] &= 0 \text{, für Skalaroperatoren $\hat f$} \\
\left[L_i,\hat f_k\right] &= -i\hbar \epsilon_{ijk}\hat f_k \text{, für Vektoroperatoren $\hat f$} \\
\left[L_i,\hat f_{kl}\right]23 &= i\hbar (\epsilon_{ikp}\delta_{nl}+\epsilon_{iln}\delta_{kp})\hat f_{pn}
\end{align*}

 
 
 
\end{multicols}
\end{document}