u\documentclass[landscape,8pt]{scrartcl}

\usepackage[left=1cm,top=1cm,right=1cm,bottom=1cm]{geometry}
\usepackage{multicol}
\usepackage{braket}
\usepackage{amsmath}
\usepackage{amssymb}




\begin{document}
\begin{multicols}{3}

\noindent Zeitabhängige Wellenfunktion:
\[
	\psi(x,t)
		= \alpha e^{-itE_0/\hbar}\ket 0 
		+ \beta  e^{-itE_1/\hbar}\ket 1  
		+ \cdots
\]
Mit den Energien $E_n = \left(n+\frac 12\right) \omega \hbar $ \\
Und den Eigenfunktionen $\ket n = \psi_n(x)$\\

\subsection{Operatoren}
\begin{tabular}{ll}
Ortsoperator 			& $\hat x = x = i \hbar \frac{\partial}{\partial p}$	\\
Impulsoperator 			& $\hat p = p = -i\hbar \frac{\partial}{\partial x}$	\\
Abstiegsoperator 		& $\hat a = \sqrt{\frac{m \omega}{2\hbar}} \hat x + \frac{i}{\sqrt{2\hbar m \omega}} \hat p$	\\
Aufstiegsoperator 		& $\hat a^\dagger = \sqrt{\frac{m \omega}{2\hbar}} \hat x - \frac{i}{\sqrt{2\hbar m \omega}} \hat p$ \\
Hamiltonoperator		& $\hat H=\frac{\hat p^2}{2m}+\frac{1}{2}m\omega^2\hat x^2 = \frac{\hat p^2}{2m} + V(\hat x)$	\\
Damit gilt				& $\hat x  = \frac 12 \sqrt{\frac{2\hbar}{m\omega}}(\hat a + \hat a^\dagger) $ \\
						& $\hat p  = \frac 1{2i} \sqrt {2\hbar m \omega} (\hat a - \hat a^\dagger) $\\
						& $\hat a\ket n = \sqrt{n}\ket{n-1} \qquad \hat a \ket 0 = 0$ \\
						& $\hat a^\dagger \ket n = \sqrt{n+1}\ket {n+1}$	\\
						& $\ket n = \frac{(\hat a^\dagger)^n}{\sqrt {n!}} \ket 0 $
\end{tabular}

\subsection{Mittelwerte}
\begin{align*}
\braket{x}=\braket{\psi|x|\psi}	&=\int_{-\infty}^{\infty} x\:|\psi(x)|^2\:dx\qquad\text{nur falls } x \text{ kein Operator!}\\
								&=\int_{-\infty}^{\infty} \psi(x)^\ast \hat x\:\psi(x)\:dx		\\
								&=\frac 12 \sqrt{\frac{2\hbar}{m\omega}}(\alpha^\ast \bra 0 + \beta^\ast \bra 1 + \dots) (\hat a + \hat a^\dagger) (\alpha \ket 0 + \beta \ket 1 + \dots) \\
\braket{p}=\braket{\psi|x|\psi}	&=\int_{-\infty}^{\infty} \hat p\:|\psi(x)|^2\:dx= \text{ analog zu}\braket{\psi|x|\psi}\text{.}	\\
\end{align*}


\subsection{Pauli-Matrizen}
\begin{align*}
S_0 &= \frac{\hbar}{2}\sigma_0 = \frac{\hbar}{2} \begin{pmatrix}1&0\\0&1\end{pmatrix}	\\
S_1 &= \frac{\hbar}{2}\sigma_1 = \frac{\hbar}{2} \begin{pmatrix}0&1\\1&0\end{pmatrix}	\\
S_2 &= \frac{\hbar}{2}\sigma_2 = \frac{\hbar}{2} \begin{pmatrix}0&-i\\i&0\end{pmatrix}	\\
S_3 &= \frac{\hbar}{2}\sigma_3 = \frac{\hbar}{2} \begin{pmatrix}1&0\\0&-1\end{pmatrix}	\\
\end{align*}
$S_1 = S_x, \; \sigma_1 = \sigma_x, \; S_2 = S_y \dots$

\end{multicols}
\end{document}